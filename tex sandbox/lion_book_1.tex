\documentclass{article}

\usepackage[utf8]{inputenc}
\usepackage[english, russian]{babel}

\title{Пробник \LaTeX}

\begin{document}
\maketitle

Освоить \LaTeX\ проще, чем \TeX. Человека, который знает систему \TeX{} и любит её, можно назвать \TeX ником. 

Полужирный шрифт начнется с \bfseries этого слова.
Снова \mdseries светлый, теперь \slshape наклонный, до нового переключения; вновь \upshape прямой.

Полужирным шрифтом набрано только {\bfseries это} слово; после скобок все идёт как прежде.

Сначала {переключим шрифт на \itshape курсив; теперь сделаем шрифт еще и {\bfseries полужирным;} посмотрите, как восстановится} шрифт после кон{ца г}руппы.

\begin{center}
	Все строки этого абзаца будут центрированы; переносов не будет, если только какое-то слово, как в уменясломалсяпробелаещеязабылсовсемпронижнееподчеркиваниеипрочиеразделителиеслионивообщесуществуют кислоте, не длинней строки.
\end{center}

\section{Начнемс}
Заголовок без звездочки.

Мойте руки.\label{wash}

\section{Продолжим}
Заголовок без звездочки.

\section*{Заголовок со звездочкой}

Оп, и нумерация пропала.

Как известно (см. раздел \ref{wash} на стр.~\pageref{wash}), руки надо мыть.

\section{Формулы}




\end{document}