\documentclass{article}

\usepackage[utf8]{inputenc}
\usepackage[english, russian]{babel}
\usepackage{amssymb,amsmath}

\title{Пробник \LaTeX}

\begin{document}
\maketitle

Освоить \LaTeX\ проще, чем \TeX. Человека, который знает систему \TeX{} и любит её, можно назвать \TeX ником. 

Полужирный шрифт начнется с \bfseries этого слова.
Снова \mdseries светлый, теперь \slshape наклонный, до нового переключения; вновь \upshape прямой.

Полужирным шрифтом набрано только {\bfseries это} слово; после скобок все идёт как прежде.

Сначала {переключим шрифт на \itshape курсив; теперь сделаем шрифт еще и {\bfseries полужирным;} посмотрите, как восстановится} шрифт после кон{ца г}руппы.

\begin{center}
	Все строки этого абзаца будут центрированы; переносов не будет, если только какое-то слово, как в уменясломалсяпробелаещеязабылсовсемпронижнееподчеркиваниеипрочиеразделителиеслионивообщесуществуют кислоте, не длинней строки.
\end{center}

\section{Начнемс}
Заголовок без звездочки.

Мойте руки.\label{wash}

\section{Продолжим}
Заголовок без звездочки.

\section*{Заголовок со звездочкой}

Оп, и нумерация пропала.

Как известно (см. раздел \ref{wash} на стр.~\pageref{wash}), руки надо мыть.

\section{Формулы}
Формула в тексте $a^2+b^2=c^2$

Выключная формула:
\[
a^2+b^2=c^2
\]

\subsection{Степени и индексы}
Катеты $a$, $b$ треугольника связаны с его гипотенузой $c$ формулой $c^2=a^2+b^2$ (теорема Пифагора).

Из теоремы Ферма следует, что уравнение
\[
x^{4357}+y^{4357}=z^{4357}
\]
не имеет решений в натуральных числах.

Обозначение $R^i_{jkl}$ для тензора кривизны было введено еще Эйнштейном.

Можно также написать $R_j{}^i{}_{kl}$.

\subsection{Дроби}
Неравенство $x+1/x\ge 2$ выполнено для всех $x>0$.
\[
\pi\approx 3{,}1415926
\]

\[
\frac{(a+b)^2}{4}-
\frac{(a-b)^2}{4}=
ab
\]

\subsection{Скобки}
\[
1+\left(\frac{1}{1-x^2}\right)
\]

\subsection{Корни}
По общепринятому соглашению, $\sqrt[3]{x^3}=x$, но $\sqrt{x^2}=|x|$.

\subsection{Штрихи и многоточия}
\[
(fg)''=f''g+2f'g'+fg''
\]

\[
{x'}^2
\]

В~детстве К.-Ф.~Гаусс придумал, как быстро найти сумму
\[
1+2+\cdots+100=5050
\]
это случилось, когда школьный учитель задал классу найти сумму чисел $1,2,\ldots,100$ 

\subsection{Имена функций}
Как знают некоторые школьники, $\log_{1/16}2=-1/4$, а $\sin(\pi/6)=1/2$.



\end{document}