\documentclass{article}

\usepackage{titlesec} % пакет для настройки заголовков; используются две функции: \titleformat и \titlespacing
\usepackage{titling}  % пакет для настройки заглавия; используется функция \theauthor
\usepackage[margin=1.25in]{geometry}

% Настройка \section
\titleformat{\section}[frame]
	{\huge}   % {format} - формат, который будет применен ко ВСЕМУ заголовку
	{}        % {label} - номер/ярлык/значок; обычно это цифры, чтобы их вернуть здесь нужно подставить \thesection
	{.25em}   % {sep} - горизонтальный разделитель между ярлыком и названием заголовка; число; обязательный аргумент
	{\filcenter\bfseries\lowercase} % {before-code} - все что между разделителем и самим заголовком

% Настройка sub section
\titleformat{\subsection}
	{\bfseries\Large}
	{\hspace{-.25in}$\bullet$}
	{0em}
	{}

% Настройка sub sub section
\titleformat{\subsubsection}[runin]
	{\bfseries}
	{}
	{0em}
	{}[---]

% Настрока отступов в sub sub section
\titlespacing{\subsubsection}
	{0em}
	{.25em}
	{1em}

% Заменяем команду своей
\renewcommand{\maketitle}{
\begin{center}
	{\huge\bfseries
	\theauthor}   % фигурные скобки нужны для разграничения действия команд;  вообще для безопасности))))

	\vspace{.25em}

	luke@lukesmith.xyz --- httt://lukesmith.xyz

\end{center}
}


\begin{document}

\title{R\'esum\'e}
\author{Luke Smith}

\maketitle

\section{Technical skills}

\subsection{Work flow}

vim, tmux or a tiling window manager. For networked devices, ssh and either git, rsync or syncthing depending on the need.

\subsection{Languages}

\subsubsection{Programming}

Lisp, Haskell, C, C++

\subsubsection{Scripting}

Python, Perl, shell, R, Praat

\subsubsection{Markup}

{\LaTeX}, HTML, CSS

\subsection{Multimedia}

\subsubsection{Image}

Most adept volleying between GIMP and imagemagick.

\subsubsection{Video}

I use Blender for video sequuencing and FFmpeg for basic operations an screencasting.

\subsubsection{Audio}

FFmpeg or Audacity for basic manipulation. Praat for more advanced or academic purpose. Also familiar with music production software like LMMS.

\section{General skills}

\subsection{Languages}

\subsubsection{Can speak}

Spanish, Chinese, Latin, some french

\subsubsection{Can read} 

Most Romance languages, Greek

\section{Job Experience}

\subsection{Univercity of Arisona}

\subsubsection{Teaching Assistant}

LING300 (Introduction to Syntax), LING150 (Language)

\subsection{Univercity of Georgia}

\subsubsection{Instructor}

LING2100 (Instructions to Linguistics)

\subsubsection{Teaching Assistant}

LING2150 (Generative Syntax)

\section{Service and Other}

\subsection{Academic}

\subsubsection{University of Arisona}

Ran an Indo-European study group.

\subsubsection{University of Georgia}

Managed the second LSUGA linguistic conference, performed video and audio recordings. Created and ran the linguistic club website. Graduate Student Mentor. Ran reading groups on Language Typology and Sociolinguistics.







\end{document}
