\documentclass[a4paper,12pt]{article} % добавить leqno в [] для нумерации слева

%%% Работа с русским языком
\usepackage{cmap}					% поиск в PDF
\usepackage{mathtext} 				% русские буквы в формулах
\usepackage[T2A]{fontenc}			% кодировка
\usepackage[utf8]{inputenc}			% кодировка исходного текста
\usepackage[english,russian]{babel}	% локализация и переносы

%%% Дополнительная работа с математикой
\usepackage{amsmath,amsfonts,amssymb,amsthm,mathtools} % AMS
\usepackage{icomma} % "Умная" запятая: $0,2$ --- число, $0, 2$ --- перечисление

% Номера формул
% \mathtoolsset{showonlyrefs=true} % Показывать номера только у тех формул, на которые есть \eqref{} в тексте.

% Шрифты
\usepackage{euscript}	 % Шрифт Евклид
\usepackage{mathrsfs}    % Красивый матшрифт

% Код с результатом сразу в тексте
\usepackage{showexpl,listings,graphicx}

% Свои команды
\DeclareMathOperator{\sgn}{\mathop{sgn}}

% Перенос знаков в формулах (по Львовскому)
\newcommand*{\hm}[1]{#1\nobreak\discretionary{}
{\hbox{$\mathsurround=0pt #1$}}{}}

% Заголовок
\author{\LaTeX{} в Вышке}
\title{1.2 Математика в \LaTeX}
\date{\today}

\begin{document} % конец преамбулы, начало документа

\maketitle

\section{Введение}

Первый            абзац.

Второй абзац.

Формула внутри строки $2 + 2 = 4$

Формула в отдельной строке: $$2 + 2 = 4$$

Чтобы формуле присвоился номер, её необходимо заключить в окружение \{equation\}:

\begin{equation}\label{eq:mrmc}
	MR=MC
\end{equation}

Вот так выглядит код формулы в окружения:
\begin{LTXexample}
\begin{equation}\label{eq:mrmc}
	MR=MC
\end{equation}
\end{LTXexample}

В первой строчке мы присвоили формуле метку --- label; ссылаться можно как на саму метку, так и на страницу, на которой она находится:

\eqref{eq:mrmc} на стр. \pageref{eq:mrmc} --- условие максимизации прибыли

\section{Нюансы работы с формулами}
\subsection{Дроби}

$$\frac{1+\dfrac{4}{2}}{6}=0.5$$

\subsection{Скобки}

$$\left(2+\frac{9}{3}\right)\times5=25$$

$$[2+3]$$

$$\{2+3\}$$

\subsection{Сдандартные функции}

$\sgn x = 1$


\subsection{Символы}

$2\times2\ne5$

$A\cap B$, $A\cup B$

\subsection{Диакритические знаки}

$\overline{x54545yz}=5$, $\tilde x=8$

\subsection{Буквы других алфавитов}

$\tg\alpha=1$

$\epsilon$, $\phi$ --- английский вариант

$\varepsilon$, $\varphi$ --- русский вариант


\section{Формулы в несколько строк}

$$
2\times 2 =4 
3\times 3 =9
$$

\subsection{Очень длинные формулы}

\begin{multline}
	1+2+3+4+5+6+7+8+9+10+11+12+\dots\\
	+50+51+52+53+54+55+56+57+58+59+60+61+52+63+64+65+\dots\\
	+96+97+98+99+100=5050	\tag{S} \label{eq:sum}
\end{multline}

\subsection{Несколько формул}

\begin{align*}
	2\times 2 &= 4 & 6\times8 &=48\\ 
	3\times 3 &= 9 & a+b &=c\\
	10\times 6545 &=65450 & 3/2 &=1.5
\end{align*}

\begin{equation}
	\begin{aligned}
		2\times 2 &= 4 & 6\times8 &=48\\ 
		3\times 3 &= 9 & a+b &=c\\
		10\times 6545 &=65450 & 3/2 &=1.5
	\end{aligned}
\end{equation}

\subsection{Системы уравнений}

$$\left\{
\begin{aligned}
	2\times x &= 4\\ 
	3\times y &= 9\\
	10\times 6545 &=z
\end{aligned}\right.
$$

$$
|x|=\begin{cases}
	x, &\text{если } x\ge0\\
	-x, &\text{если } x<0
\end{cases}
$$

\section{Матрицы}

$$
\begin{pmatrix}
	a_{11} & a_{12} & a_{13}\\
	a_{21} & a_{22} & a_{23}
\end{pmatrix}
$$

$$
\begin{vmatrix}
	a_{11} & a_{12} & a_{13}\\
	a_{21} & a_{22} & a_{23}
\end{vmatrix}
$$

$$
\begin{bmatrix}
	a_{11} & a_{12} & a_{13}\\
	a_{21} & a_{22} & a_{23}
\end{bmatrix}
$$

В уравнении \eqref{eq:sum} на стр. \pageref{eq:sum} много слагаемых

рам-пам-пам

\[ 
\begin{vmatrix} 
1 & 2\\ 
3 & 4\\ 
5 & 6 
\end{vmatrix} 
\] 




\end{document} % конец документа

